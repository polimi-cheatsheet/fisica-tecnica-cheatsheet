\section{Convezione}
Trasferimento di energia tra una superficie solida e un fluido adiacente in movimento.
\begin{description}
 \item[Convezione forzata] Avviene quando il fluido è forzato a scorrere su una superficie da mezzi esterni
 \item[Convezione naturale] Avviene quando il moto del fluido è causato da forze ascensionali che sono indotte dalle differenze di densità dovute alla variazione di temperatura del fluido in un campo gravitazionale
\end{description}

\subsection{Equazione generale}
\[ \dot{q} = h \cdot \qty(T_s - T_f) \]
\begin{align*}
h & \quad \text{coefficiente convettivo} \\
T_s & \quad \text{temperatura superficie} \\
T_f & \quad \text{temperatura fluido} \\
\end{align*}
\[ \qq*{Ponendo} R_{conv} = \frac{1}{h \cdot A} \qq{si ha che} \dot{Q}_{conv} = -\frac{\Delta T}{R_{conv}} \]

\subsection{Raggio critico di isolamento}
Considerando gusci cilindrici aventi come utimo strato uno strato di isolante soggetto a fenomeni convettivi.
\[ R_{tot} = R_{isol} + R_{conv} = \frac{\ln(\frac{R_e}{R_i})}{2\pi L k} + \frac{1}{2\pi R_e L h} \]
\begin{align*}
R_e & \quad \text{raggio esterno isolante} \\
R_i & \quad \text{raggio interno isolante}
\end{align*}
Si ha che inizialmente $R_{tot}$ decresce all'aumentare dello spessore di isolante.
Si ha $R_{tot}$ minimo quando $R_e = R_{critico} = \frac{k}{h}$.
