\documentclass[10pt,landscape]{article}
\usepackage[italian]{babel}
\usepackage[utf8]{inputenc}
\usepackage{multicol}
\usepackage{calc}
\usepackage{ifthen}
\usepackage[landscape]{geometry}
\usepackage{hyperref}
\usepackage{amsmath}
\usepackage{amssymb}
\usepackage{tabularx}
\usepackage{caption}
\usepackage{verbatim}
\usepackage{systeme}
\usepackage{nicefrac}
\usepackage{accents}
\usepackage{enumitem}
\usepackage[printwatermark]{xwatermark}
\usepackage{tikz}
\usetikzlibrary{calc,matrix,arrows,intersections}
\usepackage[compact]{titlesec}
\usepackage{microtype}
\usepackage[flushleft]{threeparttable}
\usepackage{textcomp}
\usepackage{pifont}
\usepackage{pgfplots}
\usepackage{physics}
\usepackage{siunitx}
\usepackage{booktabs}
\usepackage{stmaryrd}
\usepackage{chemformula}

% This sets page margins to .5 inch if using letter paper, and to 1cm
% if using A4 paper. (This probably isn't strictly necessary.)
% If using another size paper, use default 1cm margins.
\ifthenelse{\lengthtest { \paperwidth = 11in}}
{ \geometry{top=.5in,left=.5in,right=.5in,bottom=.5in} }
{\ifthenelse{ \lengthtest{ \paperwidth = 297mm}}
	{\geometry{top=1cm,left=1cm,right=1cm,bottom=1cm} }
	{\geometry{top=1cm,left=1cm,right=1cm,bottom=1cm} }
}

% Turn off header and footer
\pagestyle{empty}

% Reduce size of \section e \subsection
\titleformat{\section}{\normalfont\large\bfseries}{\thesection}{1em}{}
\titleformat{\subsection}{\normalfont\normalsize\bfseries}{\thesubsection}{1em}{}
\titlespacing{\section}{0pt}{0ex}{-0.5ex}
\titlespacing{\subsection}{0pt}{0ex}{-0.5ex}

% Define BibTeX command
\def\BibTeX{{\rm B\kern-.05em{\sc i\kern-.025em b}\kern-.08em
		T\kern-.1667em\lower.7ex\hbox{E}\kern-.125emX}}

% Don't print section numbers
\setcounter{secnumdepth}{0}

\setlength{\parindent}{0pt}
\setlength{\parskip}{0pt plus 0.5ex}

\setlist[itemize]{noitemsep, nolistsep}

% No idea, without this tikz gives a warning
\pgfplotsset{compat=1.16}

\pgfmathdeclarefunction{gauss}{2}{%
  \pgfmathparse{1/(#2*sqrt(2*pi))*exp(-((x-#1)^2)/(2*#2^2))}%
}

\newcommand{\sla}{\shortleftarrow}
\newcommand{\sra}{\shortrightarrow}

\newcommand{\gaussiana}[4]{
	\begin{axis}[
		hide axis,
		axis lines=left,
		xtick=\empty,
		ytick=\empty,
		every axis plot post/.append style={mark=none,domain=-1:1,samples=50,smooth},
		xmin=#1,
		xmax=#2,
		ymin=#3,
		ymax=#4,
		width=5cm
	]
		\addplot[black] {gauss(0,0.6)};
	\end{axis}
}
