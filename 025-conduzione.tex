\section{Conduzione}
Trasferimento di energia per effetto dell'interazione delle particelle di una sostanza dotata di maggiore energia con quelle adiacenti dotate di minore energia.
Può avvenire nei solidi, liquidi e gas.
\subsection{Postulato di Fourier}
\[ \vec{q} = - k \gradient{T} \]

\subsection{Equazione di Fourier}
\[ \frac{\rho c_v}{k} \pdv{T}{t} = \laplacian{T} + \frac{\sigma}{k} \]
\begin{align*}
\vec{q} & \quad \text{vettore flusso di calore areico} \\
k & \quad \text{conduttività termica} \\
\sigma & \quad \text{potenza generata nell'unità di volume} \\
\rho = \frac{1}{v} & \quad \text{massa volumica} \\
\end{align*}

\subsection{Casi particolari equazione di Fourier}
\begin{align*}
\laplacian{T} = \frac{\rho c_v}{k} \pdv{T}{t} & \quad \text{Assenza generazione potenza} \\
\laplacian{T} + \frac{\sigma}{k} = 0 & \quad \text{Regime strazionario (eq. Poisson)}\\
\laplacian{T} = 0 & \quad \text{Assenza generazione e regime stazionario}\\
\end{align*}

\subsection{Coordinate cartesiane}
\[ \pdv[2]{T}{x} + \pdv[2]{T}{y} + \pdv[2]{T}{z} + \frac{\sigma(x,y,z,t)}{k} = \frac{\rho c_v}{k}\pdv{T}{t} \]
\subsubsection{Parete piana infinita in $y-z$ in regime stazionario}
\[ T = -\frac{\sigma}{2k}x^2 + Ax + B \qquad \dot{q} = \sigma x - Ak \]

\subsubsection{Lastra piana senza generazione di potenza}
Avendo $T=T_1$ in $x=0$ e $T = T_2$ in $x=s$ (con $s$ lo spessore della lastra piana) si ha che:
\[ T = \frac{T_2 - T_1}{s}x + T_1 \qquad \dot{q} = -k \left(\frac{T_2 - T_1}{s} \right) \]
\[\qq*{Ponendo} R_c = \frac{s}{A \cdot k} \qq{si ha che} \dot{Q}_{cond} = -\frac{\Delta T}{R_c}\]

\subsection{Coordinate cilindriche}
\[ \pdv[2]{T}{r} + \frac{1}{r}\pdv{T}{r} + \frac{1}{r^2}\pdv[2]{T}{\varphi} + \pdv[2]{T}{z} + \frac{\sigma(r,\varphi,z,t)}{k} = \frac{\rho c_v}{k} \pdv{T}{t} \]

\subsubsection{Cilindro pieno o cavo di altezza infinita}
Supponendo $T=T(r)$ si ha che, in regime stazionario:
\[T = -\frac{\sigma}{4k}r^2 + A\ln(\frac{r}{B}) = -\frac{\sigma}{4k}r^2 + A\ln(r) + C \]
\[\dot{q} = -k \pdv{T}{r} = \frac{\sigma}{2}r - \frac{k}{r}A \]

\subsubsection{Barra cilindrica piena con generazione di potenza}
Ponendo come condizione $\pdv{T}{r} = 0$ per $r = 0$ e $T = T_2$ per $r = R$ si ha che:
\[T = \frac{\sigma}{4k}\qty(R^2-r^2) + T_2 \qquad \dot{q}_{\text{areico}} = \frac{\sigma}{2}r \]
\[\dot{q}_{\text{per unità di lunghezza}} = \pi r^2 \sigma \]

\subsubsection{Cilindro cavo senza generazione di potenza}
Ponendo $T = T_i$ per $r = R_i$ e $T = T_e$ per $r = R_e$ si ottiene
\[T = T_i + \frac{T_e - T_i}{\ln(\frac{R_e}{R_i})}\ln(\frac{r}{R_i}) \qquad \dot{q}_{\text{areico}} = k \frac{T_i-T_e}{\ln(\frac{R_e}{R_i})}\frac{1}{r}\]
\[\dot{q}_{\text{per unità di lunghezza}} = \frac{2\pi k}{\ln(\frac{R_e}{R_i})}\qty(T_i - T_e) \]
\[\qq*{Ponendo} R_{cil} = \frac{\ln(\frac{R_e}{R_i})}{2\pi L k} \qq{si ha che} \dot{Q} = \frac{T_i-T_e}{R_{cil}}\]

\subsection{Coordinate sferiche}
\begin{align*}
    & \pdv[2]{T}{r} + \frac{2}{r}\pdv{T}{r} + \frac{1}{r^2}\pdv[2]{T}{\theta} + \frac{\cot(\theta)}{r^2}\pdv{T}{\theta} + \frac{1}{r^2 \sin^2(\theta)}\pdv[2]{T}{\varphi} + \frac{\sigma}{k} =\\
    & = \frac{\rho c_v}{k} \pdv{T}{t}
\end{align*}

\subsubsection{Sfera piena o cava}
Supponendo $T = T(r)$ e regime stazionario si ha che
\[ T = -\frac{\sigma}{6k}r^2 + \frac{A}{r} + B \]
Analogamente al caso cilindrico si ha che:
\[ R_{sfera} = \frac{R_2 - R_1}{4\pi R_1 R_2 k} \qquad \dot{Q} = \frac{T_i-T_e}{R_{sfera}} \]
