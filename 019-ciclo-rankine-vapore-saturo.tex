\section{Ciclo Rankine a vapore saturo}
\begin{tikzpicture}[thick,>=stealth']
    \coordinate (O) at (0,0);
    \draw[->] (0,0) -- (3.5,0) coordinate[label = {below:$s$}] (xmax);
    \draw[->] (0,0) -- (0,3.5) coordinate[label = {left:$T$}] (ymax);

    \gaussiana{-1.2}{1.2}{0}{0.7}

    \draw (0.54,1) node[below] {1} -- (0.54,1.3) node[left]{2} -- (0.98,1.8) node[above left] {3} -- (2.42,1.8) node [above right] {4} -- (2.42,1) node [below right] {5} -- cycle;

\end{tikzpicture}

\begin{tabular}{p{2.5cm}l}
Trasformazione 1-2: & compressione isoentropica (pompa)\\
Trasformazione 2-4: & riscaldamento isobaro (bruciatore)\\
Trasformazione 4-5: & espansione isoentropica (turbina)\\
Trasformazione 5-1: & raffreddamento isobaro (condensatore)\\
\end{tabular}

Rendimento ciclo Rankine a vapore saturo:
\[ \eta = 1 - \frac{\dot{Q}_F}{\dot{Q}_C} = 1 - \frac{\dot{m}(h_5-h_1)}{\dot{m}(h_4-h_2)} = 1 - \frac{h_5-h_1}{h_4-h_2} \]

Per il calcolo del salto entalpico in pompa essendo $\dd{h} = T\dd{s} + v\dd{P}$ ma $\dd{s} = 0$ e fluido incomprimibile:
\[ h_2-h_1 = v(P_2-P_1) \]

Il ciclo Rankine a vapore saturo non viene utilizzato perchè all'uscita dalla turbina il titolo non è sufficientemente alto (si vorrebbe avere titolo superiore a 0.9).
