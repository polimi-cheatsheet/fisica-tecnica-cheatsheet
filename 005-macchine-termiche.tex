\section{Macchina termodinamica}
La macchina termodinamica è un sistema termodinamico composto ed isolato che nel caso più semplice è realizzato da
\begin{itemize}
\item due serbatoi di calore
\item un serbatoio di lavoro
\item una macchina ciclica che è in grado di interagire con continuità con i serbatoi di calore e lavoro
\end{itemize}

\begin{description}
\item[Serbatoio di calore]
sistema termodinamico che scambia solo calore con l'esterno senza alterare il suo stato interno;
gli scambi avvengono con trasformazioni quasi-statiche internamente reversibili.
\item[Serbatoio di lavoro]
sistema termodinamico che scambia solo lavoro con l'esterno senza alterare il suo stato interno;
gli scambi avvengono con trasformazioni quasi-statiche internamente reversibili.
\end{description}

\textbf{Risoluzione problemi macchine termiche}\\
Bisogna impostare e risolvere il sistema contente le equazioni di bilancio
\[
    \begin{cases}
    \Delta U_Z = 0 \\
    \Delta S_Z = S_{irr}
    \end{cases}
\]

\subsection{Rendimenti macchine termiche}
{\renewcommand{\arraystretch}{1.5}
\begin{tabular}{p{4cm}c}
\textbf{Macchina motrice} & $\eta = \frac{L}{Q_C}$ \\
→ con serbatoi a massa infinita & $\eta = 1 - \frac{T_F}{T_C} - \frac{T_F}{Q_C}S_{irr}$ \\
\phantom{→}→ reversibile ($S_{irr} = 0$) & $\eta = 1 - \frac{T_F}{T_C}$ \\
\\
\textbf{Macchina operatrice} & (serbatoi a temp. cost.) \\
→ frigorifera & $\epsilon_f = \frac{Q_F}{L}$ \\
\phantom{→}→ reversibile & $\epsilon_{f,rev} = \frac{T_F}{T_C - T_F}$ \\
→ pompa di calore & $\epsilon_{pdc} = \frac{Q_C}{L}$ \\
\phantom{→}→ reversibile & $\epsilon_{pdc,rev} = \frac{T_C}{T_C - T_F}$ \\
\end{tabular}
}
