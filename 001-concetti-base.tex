\section{Concetti base}
\subsection{Sistema termodinamico}
\begin{tabular}{lccc}
    \toprule
    & Calore & Lavoro & Massa\\ \midrule
    Adiabatico & no & & \\
    Diatermano & sì & & \\
    Rigido & & no & \\
    Deformabile & & sì & \\
    Chiuso (impermeabile) & & & no \\
    Aperto (permeabile) & & & sì \\
    Isolato & no & no & no \\
    \bottomrule
\end{tabular}

\subsection{Grandezze}
\begin{itemize}
    \item \textbf{Intensive}: non dipendono dall'estensione del sistema (temperatura, pressione, densità);
    \item \textbf{Estensive}: dipendono dall'estensione del sistema (massa, volume);
    \item \textbf{Estensive specifiche}: estensiva/estensiva ($v = \frac{V}{M}$).
\end{itemize}

\subsection{Legge di Duhem}
In un sistema semplice monocomponente il numero di parametri intensivi o estensivi specifici
necessari per descriverlo all'equilibrio è 2.


\subsection{Regola di Gibbs}
\[V = C + 2 - F\]

\begin{tabular}{ll}
    $V$ & numero di variabili intensive indipendenti \\
    $C$ & numero di componenti \\
    $F$ & numero di fasi \\
\end{tabular}

Ne deriva l'esistena di una legge di stato:
\[f(P, v, T) = 0\]

\subsection{Trasformazioni termodinamiche}
\begin{tabular}{p{3.1cm}p{4cm}}
    \toprule
    Trasformazione & Caratteristiche \\
    \midrule
    Quasi-statica o \newline internamente reversibile & Costituita da una successione di stati di equilibrio; può non essere reversibile \\
    Reversibile & Se percorsa in senso inverso, riporta sistema e ambiente nello stato iniziale \\
    Irreversibile & Non è rappresentabile su un diagramma di stato \\
    Chiusa o ciclica & Gli estremi della trasformazione coincidono \\
    Elementare & Se una delle grandezze di stato si mantiene costante durante la trasformazione \\
    \bottomrule
\end{tabular}

\subsection{Gas ideali}
Un gas può essere considerato ideale per pressioni minori di $\si{10\bar}$.

\[PV = nRT\]
\begin{tabular}{ll}
    $P$ & pressione $[\si{\pascal}]$ \\
    $V$ & volume $[\si{m^3}]$ \\
    $n$ & numero di moli $[\si{\mol}]$ \\
    $R$ & costante universale dei gas ideali $\SI{8314}{\J/k\mol.\K}$ \\
    $T$ & temperatura $[\si{\K}]$
\end{tabular}

\[Pv = \frac{R}{M_m}T = R^*T \]
\begin{tabular}{ll}
    $M_m$ & massa molecolare \\
\end{tabular}

\subsection{Gas reali}
Equazione di van der Waals:
\[\left(P + \frac{a}{v_m^2}\right)(v_m-b) = RT\]

\subsection{Liquidi e solidi}
\begin{align*}
    \dd{v} &= \left(\pdv{v}{T}\right)_P \dd{T} + \left(\pdv{v}{P}\right)_T \dd{P} \\
    \dd{v} &= \beta v \dd{T} - K_T v \dd{P}
\end{align*}

\begin{align*}
    \beta = \frac{1}{v} \left(\pdv{v}{T}\right)_P & \quad \text{coeff. di dilatazione termica isobaro} \\
    K_T = -\frac{1}{v} \left(\pdv{v}{P}\right)_T & \quad \text{coeff. di dilatazione termica isotermo} \\
\end{align*}
