\section{Soluzioni per aumento rendimento ciclo Rankine}

\begin{itemize}
    \item Abbassamento della pressione di condensazione $\Rightarrow$ maggior lavoro prodotto (ma anche più calore richiesto all'uscita della pompa, titolo minore in uscita dalla turbina e rischio di infiltrazioni se pressione di condensazione minore pressione atmosferica)
    \item Aumento della temperatura finale di surriscaldamento $\Rightarrow$ aumento di lavoro prodotto (ma anche aumento calore richiesto) e titolo in uscita dalla turbina più alto.
    \item Aumento della pressione di vaporizzazione (a parità di temperatura massima) $\Rightarrow$ stesso lavoro e meno calore in uscita durante condensazione (ma anche titolo minore in uscita da turbina)
    \item Surriscaldamenti ripetuti con espansioni in più stadi $\Rightarrow$ permettono di aumentare lavoro e titolo in uscita da turbina
    \item Rigenerazione $\Rightarrow$ si estrae vapore dalla turbina e si mette a contatto con il liquido a bassa temperatura
    \item Cogenerazione $\Rightarrow$ si utilizza vapore in uscita da turbina per teleriscaldamento o per altri scopi
\end{itemize}
